\documentclass[a4paper,12pt]{article}
\usepackage[utf8]{inputenc}
\newcommand{\myparagraph}[1]{\paragraph{#1}\mbox{}\\}
\usepackage[a4paper,margin=3.5cm]{geometry}
\usepackage{url}
\usepackage{dirtytalk}
\usepackage{graphicx}
\usepackage{wrapfig}
\usepackage[T1]{fontenc}
\setlength{\parskip}{1em}
\usepackage{amssymb}
\usepackage{amsmath}
\usepackage{tcolorbox}
\usepackage{mathtools}
\usepackage{enumitem}
\usepackage{amsthm}


% Lets you use \blankpage to make a blank page
\newcommand{\blankpage}{
\newpage
\thispagestyle{empty}
\mbox{}
\newpage
}

% Define a slanted theorem style
\newtheoremstyle{sltheorem}
{}                % Space above
{}                % Space below
{}                % Theorem body font % (default is "\upshape")
{}                % Indent amount
{\bfseries}       % Theorem head font % (default is \mdseries)
{.}               % Punctuation after theorem head % default: no punctuation
{ }               % Space after theorem head
{}                % Theorem head spec
\theoremstyle{sltheorem}


% Theorems
\newtheorem{theorem}{Theorem}[section]
\newtheorem{definition}[theorem]{Definition}
\newtheorem{corollary}[theorem]{Corollary}
\newtheorem{lemma}[theorem]{Lemma}
\newtheorem{example}[theorem]{Example}
\newtheorem{remark}[theorem]{Remark}
\newtheorem{fact}[theorem]{Fact}


% List stuff
\setlist[enumerate]{topsep=0pt}
\setlist[itemize]{topsep=0pt}

% Title stuff
\title{MATH500}
\date{Spring 2024}
\author{Matthew Leonardson}

\begin{document}
\maketitle

\section{January 17, 2024}

\begin{definition}\label{group_def}
    A \textit{group} is a set $G$ with a binary operation such that 
    \begin{enumerate}
        \item $(xy)z = x(yz)$ for all $x, y, z \in G$.
        \item There exists $e \in G$, the identity.
        \item For all $x \in G$ there exists $x^{-1}$ such that $x x^{-1} = e = x^{-1} x$. 
    \end{enumerate}
    Further, a group is \textit{abelian} if \begin{enumerate}[resume]
        \item $xy = yx$ for all $x, y \in G$.
    \end{enumerate}
\end{definition}

\begin{definition}
    A \textit{monoid} is a set $M$ and a binary operation that only satisfy the first two axioms of~\ref{group_def}.
\end{definition}

\begin{example}
    The following are examples of groups
    \begin{itemize}
        \item $C_n$: the cyclic group of order $n$. Written multiplicatively.
        \item $\mathbb{Z}/n$: the integers modulo $n$. Identical to $C_n$, but written additively.
        \item $D_{2n}$: the dihedral group\footnote{Some authors use $D_n$ for the dihedral group of order $2n$.} of order $2n$. Defined in~\ref{dihedral_def}.
        \item $S_n$: the symmetric group of degree $n$. All permuations of $n$ numbers with the group operation being function composition.
        \item $GL_n(k)$: the general linear group of degree $n$. All invertible $n \times n$ matrices over a field $k$.
        \item $Q_8$: the quaternion group. Defined in~\ref{quaternion_def}.
    \end{itemize}

\end{example}

\begin{definition}\label{dihedral_def}
    The \textit{dihedral group} of order $2n$ is the group of rotational symmetries of a regular $n$-gon in 3D space. More abstractly, it is a group with elements $\{r, s\}$ such that $r^n = s^2 = e$ and $rs = sr^{-1}$.
\end{definition}

\begin{remark}
Bridging these two interpretations of the dihedral group, we can think of $r$ as being a rotation of the $n$-gon and $s$ as being a flipping of the $n$-gon.
\end{remark}

\begin{definition}\label{quaternion_def}
The \textit{quaternion group} is the set $\{\pm 1, \pm i, \pm j, \pm k\}$ and multiplication defined such that ${(-1)}^2 = 1$ and $i^2 = j^2 = k^2 = ijk = -1$.
\end{definition}

\begin{definition}
    Given a group $G$ and subset $H$, we say $H$ is a \textit{subgroup} if 
    \begin{enumerate}
        \item $H$ is not empty.\footnote{This is equivalent to $e \in H$.}
        \item $x \in H$ implies $x^{-1} \in H$.
        \item $x, y \in H$ implies $xy \in H$. 
    \end{enumerate}
\end{definition}

\begin{definition}
    For a group $G$ and $S \subseteq G$, the subgroup \textit{generated} by $S$ is \[\langle S \rangle = \bigcap_{\substack{H \leq G \\ S \subseteq H}} H.\]
\end{definition}

\begin{fact}
    For a group $G$ and $S \subseteq G$, $\langle S \rangle$ is a subgroup of $G$.
\end{fact}

\begin{definition}\label{word_def}
    Given a group $G$ and $S \subseteq G$, a \textit{word} in $S$ is $g \in G$ written $g = g_1 g_2 \dots g_n$ where $g_i \in S$ or $g_i^{-1} \in S$. 
\end{definition}

\begin{fact}
    For a group $G$ and $S \subseteq G$, the set of words in $S$ is $\langle S \rangle$.
\end{fact}

\begin{definition}
    A group $G$ is cylic if there exists $a \in G$ such that $G = \langle a \rangle$.\footnote{This is abuse of notation, as we should write $\langle \{a\} \rangle$. However, this is rarely done.} 
\end{definition}




\end{document}
